\chapter{Conclusiones}

En este trabajo se introdujo una nueva forma de solución para el problema \textit{3-D PSP}. La estrategia de solución implementa un algoritmo memético que incorpora, en el proceso de búsqueda, la información extraída de la \textit{Protein Data Bank (PDB)} en forma de Lista de Probabilidad de Ángulos (APL). Los resultados mostrados por el algoritmo propuesto muestran que puede encontrar buenas soluciones en términos de energía (valores negativos) y para algunos casos RMSD (menores a $2.6\AA$) cuando son comparadas con la estructura experimental. Adicionalmente, los resultados de los experimentos computacionales muestran la contribución positiva de las APLs al algoritmo, ya que conduce al MA y le permite obtener soluciones similares a las experimentales a nivel visual, no asi las que no usan APLs llegando a RMSDs de $>5\AA$. Esto se repite al evaluar el MA con proteínas extensas, si bien los resultados son bastante mayores que las proteínas en estudio, se debe recalcar que este problema es NP-Completo y el enfoque usado es de simulación \textit{ab initio}, lo que implica que cada vez que se evalúa una secuencia extensa implica un tiempo elevado, que repercute en la cantidad de generaciones, y en consecuencia, su convergencia a una solución.

Respecto a los problemas detectados a lo largo de la experimentación, se comprobó lo mencionado en \citealp{hornak:2006} sobre las falencias que presenta los campos de fuerza AMBER, resaltando la parametrización que presenta para la formación de hojas-$\beta$. Esto queda en evidencia al revisar que el MA no logra formar correctamente las hojas-$\beta$ y presenta problemas que repercuten en la conformación de las zonas en las que se producen peglamientos tipo \textit{Turn} o \textit{Coil}.

Sobre las herramientas para el desarrollo de \textit{software}, se debe mencionar la complejidad y nula flexibilidad del lenguaje de programación \textit{NAB} provisto por la \textit{suite AmberTools 14} para Modelamiento Molecular. Este problema causo en varias ocasiones cuellos de botella desde el punto de vista de la implementación. Además, esto presenta una desventaja para un plan de experimentación más amplio debido a que solo posee los campos de fuerza de \textit{Amber} y no permite probar otros como \textit{Charmm} o \textit{Rosetta}. Como solución, se podría usar el lenguaje Python con su bibliotecas para Modelamiento Molecular llamadas \textit{The Molecular Modelling Toolkit (MMTK)}, \textit{SIMTK} o \textit{PyOpenMM}.

Luego de la etapa de experimentación se desprenden interesantes aristas que podrían ser abarcadas en trabajos futuros. Medidas que pueden mejorar este trabajo parten por extender los histogramas que dan origen a las APLs usadas, se propone generar histogramas para residuos más específicos que consideren, además de la estructura secundaria, al residuo adyacente; lo que daría una cantidad de $8_{\text{e. secundaria}}{\times}20_{residuos}{\times}20_{residuos}=3200_{APL}$ y permitiría reducir aún más el espacio de búsqueda. Otra mejora que se plantea, es detectar aquellas zonas de la secuencia que se repiten a nivel de la población de agentes, para preservarlas y aplicar mayor esfuerzo computacional en aquellas zonas que son complejas de predecir. Como acotación final, cambiar el lenguaje de programación NAB permitiría usar los campos de fuerzas antes mencionados e incluso su combinación en el cálculo de energía, por ejemplo, podría usarse \textit{Rosseta} para las hojas-$\beta$ y \textit{Amber} para las hélices-$\alpha$ (ya que logra obtener RMSDs bajos menores a $2\AA$).

Finalmente, los resultados obtenidos fueron aceptados para su exposición en la \textit{Genetic and Evolutionary Computation Conference} (GECCO 2015) a efectuarse en Madrid, España los días 11 al 15 de Julio de 2015.